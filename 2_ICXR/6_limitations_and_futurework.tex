\section{LIMITATIONS AND FUTURE WORK}
Currently, the deployment of RIO and VIO in the VR twin system remains semi-automated, requiring manual adjustments to ensure precise alignment. During the course of this study, limitations in the VR headset's functionality made it difficult to capture camera images for automatic positioning. Relying solely on sensor technology for precise positioning is costly and may introduce additional base station deployments, thereby increasing system complexity. Future research will explore fully automated deployment methods, enabling precise RIO positioning and real-time VIO reconstruction as soon as the user dons the VR headset. By integrating advanced computer vision techniques with machine learning algorithms, dynamic prediction and adjustment of alignment errors will be achieved, facilitating spatial alignment without human intervention.

Another limitation is that current system supports only single-user interaction, whereas multi-user scenario (e.g., group experiments or team problem-solving activities) require synchronized interaction and shared virtual environments. Future research will investigate the extension of multi-user interaction, focusing on the development of shared virtual environments that support collaborative activities. Additionally, the potential educational value of VRTI will be explored, particularly in the context of collaborative learning modes in VR.

In addition, user feedback has highlighted issues with the durability of haptic feedback devices and the physical fatigue associated with prolonged interaction, which may impact the system's long-term usability and user satisfaction. Future research will prioritize enhancing the durability of interaction devices. Advanced materials will be employed to improve the lifespan and responsiveness of haptic feedback devices. Concurrently, ergonomic design principles will be incorporated to reduce user interaction burden.

Finally, current system's experimental setup and user interaction rely on predefined rules, limiting its adaptability to diverse learning scenario and personalized learning needs. Future research will focus on integrating AI into the VRTI to analyze user's behavior, predict learning requirements, and dynamically adjust the experimental environment. For instance, AI could guide users through complex experiments, provide personalized feedback, and recommend additional learning resources based on progress. Furthermore, AI-driven data analysis could be employed to assess learning outcomes and inform improvements to both the system and the curriculum.