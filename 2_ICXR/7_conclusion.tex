\section{CONCLUSION}
This paper addresses the lack of haptic feedback in Gesture Interaction (GI) within VR immersive learning environments by proposing a Virtual-Real Twin Interaction (VRTI). For the momentum conservation experiment in physics learning at high school, three types of Virtual-Real twin (VR twin) were designed and implemented, supporting grasping, pressing, and pinching hand manipulations. Comparative evaluations between VRTI and GI in the momentum conservation experiment scenario demonstrate that the former significantly enhances user learning motivation and immersion without significantly increasing cognitive load. Moreover, it better aids users in understanding experimental content and improves their ability to apply knowledge comprehensively. Future research will explore the application of VR twins in other physics learning scenarios, improve hardware structures, and optimize interaction methods to enhance user learning experiences and operational comfort.