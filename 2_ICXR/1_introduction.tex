\section{INTRODUCTION}
With the rapid advancement of VR technology, immersive learning environments have become a focal point of research and are widely applied in fields such as entertainment, education, and healthcare \cite{luo2020dream,yeung2021virtual}. Introducing immersive learning environments into physics experiment teaching can address the challenges faced by traditional methods, such as difficulties in understanding, high costs, cumbersome operations, and limited resources \cite{yang2007impact,abu2018design}. By transforming traditional teaching models, immersive learning environments create more interactive and engaging experiences, thereby stimulating students' interest and motivation. For instance, studies by Dalgarno \& Lee demonstrate that immersive learning environments significantly enhance students' ability to understand complex concepts and promote deep learning \cite{dalgarno2010learning}. Similarly, Campos et al. utilized immersive learning environments in introductory physics courses to teach vector concepts, allowing students to manipulate vectors in a 3D grid, identify their angular components, and measure their magnitudes. The results highlight the potential of this approach in fostering abstract conceptualization and operational skills, providing more intuitive explanations of physics concepts \cite{campos2022impact}.

Immersive learning environments offer new possibilities for physics experiment teaching by simulating experimental phenomena through visual and auditory means, reducing equipment requirements, and increasing experimental flexibility. However, the predominantly visual interaction lacks the haptic feedback present in real physical experiments, making it difficult for traditional immersive learning environments to provide an authentic physical operation experience \cite{giri2021application}. Learners in virtual environments can only observe physical phenomena but cannot feel the actual touch at play through haptic feedback, resulting in an incomplete learning experience. Physics experiments are not merely about observing phenomena but also involve direct perception of the interactions between forces and objects. Therefore, integrating haptic feedback into immersive physics learning environments is crucial. By incorporating haptic feedback, students can perceive the weight, resistance, and elasticity of objects in virtual environments, gaining a more intuitive operational experience and deepening their understanding of physics concepts and laws \cite{minaker2016handson}. Nevertheless, traditional haptic devices often provide only basic force feedback and may constrain learners' interactive actions, making it challenging to meet the demands of complex hand operations in physics experiments \cite{bonfert2023challenges}. For example, in real physics experiments, hand operations encompass a variety of actions, including grasping, pressing, pinching, rotating, and dragging. These operations not only require precise control of hand movements but also demand perception and adjustment of force magnitude, direction, and application, which are critical to the success of the experiment. However, existing haptic devices cannot accurately simulate these hand operations, resulting in a significant disparity between virtual experimentation and real-world operational fidelity.

In summary, this study makes the following key contributions.

\chadded[id=zheliku]{
  % 原文贡献点描述替换为:
  First, we propose Virtual-Real Twin Interaction (VRTI), a novel framework that extends passive haptic proxies by introducing real-time spatiotemporal synchronization between physical objects and virtual counterparts. Unlike prior static props, VRTI enables dynamic force feedback through sensor-actuator synchronization while maintaining precise virtual-real alignment during complex manipulations.
}

% First, we propose Virtual-Real Twin Interaction (VRTI), which integrates hand tracking with realistic haptic feedback into VR physics learning environments. The VRTI system comprises Virtual-Real twins (VR twins): a Virtual Interactive Object (VIO) for visual feedback and a Real Interactive Object (RIO) for haptic feedback. The RIO is fabricated using 3D printing technology based on 3D modeling designs, while the VIO is rendered in the virtual environment. Users directly manipulate these VR twins with both hands, achieving real-time synchronized visual and haptic feedback that delivers a highly realistic and natural interaction experience. This approach is characterized by high immersion, realistic haptic feedback, natural interaction, and low cost.

Second, we introduce a virtual-real alignment method for VRTI to avoid penetration issues between hands and VR twins. Position and orientation errors in hand tracking often cause penetration issues between the virtual hand and VIO. Our method addresses this visuo-haptic misalignment through gesture prediction optimization, eliminating penetration while maintaining interaction fidelity.

Finally, we validate the advantages of VRTI through an educational experiment. Collaborating with physics professors and researchers at XXX
% Beijing Normal 
University, we designed a momentum conservation experiment aligned with high school physics curricula. A controlled comparison between VRTI ($N=32$) and GI ($N=32$) demonstrates that VRTI:

\begin{enumerate}
  \item Does not significantly increase cognitive load ($p = 0.602$).

  \item Significantly enhances learning motivation ($p < 0.001$) and immersion ($p < 0.001$).

  \item Improves comprehension of experimental content ($p = 0.05$) and knowledge application capabilities ($p = 0.005$) more effectively than GI alone.
\end{enumerate}